\documentclass[
  doc,
  floatsintext,
  longtable,
  a4paper,
  nolmodern,
  notxfonts,
  notimes,
  colorlinks=true,linkcolor=blue,citecolor=blue,urlcolor=blue]{apa7}

\usepackage{amsmath}
\usepackage{amssymb}

\geometry{inner=1in, outer=1in}
\fancyhfoffset[LE,RO]{0cm}


\usepackage[bidi=default]{babel}
\babelprovide[main,import]{english}


\babelfont{rm}[,RawFeature={fallback=mainfontfallback}]{CMU Serif}
% get rid of language-specific shorthands (see #6817):
\let\LanguageShortHands\languageshorthands
\def\languageshorthands#1{}

\RequirePackage{longtable}
\RequirePackage{threeparttablex}

\makeatletter
\renewcommand{\paragraph}{\@startsection{paragraph}{4}{\parindent}%
	{0\baselineskip \@plus 0.2ex \@minus 0.2ex}%
	{-.5em}%
	{\normalfont\normalsize\bfseries\typesectitle}}

\renewcommand{\subparagraph}[1]{\@startsection{subparagraph}{5}{0.5em}%
	{0\baselineskip \@plus 0.2ex \@minus 0.2ex}%
	{-\z@\relax}%
	{\normalfont\normalsize\bfseries\itshape\hspace{\parindent}{#1}\textit{\addperi}}{\relax}}
\makeatother




\usepackage{longtable, booktabs, multirow, multicol, colortbl, hhline, caption, array, float, xpatch}
\setcounter{topnumber}{2}
\setcounter{bottomnumber}{2}
\setcounter{totalnumber}{4}
\renewcommand{\topfraction}{0.85}
\renewcommand{\bottomfraction}{0.85}
\renewcommand{\textfraction}{0.15}
\renewcommand{\floatpagefraction}{0.7}

\usepackage{tcolorbox}
\tcbuselibrary{listings,theorems, breakable, skins}
\usepackage{fontawesome5}

\definecolor{quarto-callout-color}{HTML}{909090}
\definecolor{quarto-callout-note-color}{HTML}{0758E5}
\definecolor{quarto-callout-important-color}{HTML}{CC1914}
\definecolor{quarto-callout-warning-color}{HTML}{EB9113}
\definecolor{quarto-callout-tip-color}{HTML}{00A047}
\definecolor{quarto-callout-caution-color}{HTML}{FC5300}
\definecolor{quarto-callout-color-frame}{HTML}{ACACAC}
\definecolor{quarto-callout-note-color-frame}{HTML}{4582EC}
\definecolor{quarto-callout-important-color-frame}{HTML}{D9534F}
\definecolor{quarto-callout-warning-color-frame}{HTML}{F0AD4E}
\definecolor{quarto-callout-tip-color-frame}{HTML}{02B875}
\definecolor{quarto-callout-caution-color-frame}{HTML}{FD7E14}

%\newlength\Oldarrayrulewidth
%\newlength\Oldtabcolsep


\usepackage{hyperref}



\usepackage{color}
\usepackage{fancyvrb}
\newcommand{\VerbBar}{|}
\newcommand{\VERB}{\Verb[commandchars=\\\{\}]}
\DefineVerbatimEnvironment{Highlighting}{Verbatim}{commandchars=\\\{\}}
% Add ',fontsize=\small' for more characters per line
\usepackage{framed}
\definecolor{shadecolor}{RGB}{241,243,245}
\newenvironment{Shaded}{\begin{snugshade}}{\end{snugshade}}
\newcommand{\AlertTok}[1]{\textcolor[rgb]{0.68,0.00,0.00}{#1}}
\newcommand{\AnnotationTok}[1]{\textcolor[rgb]{0.37,0.37,0.37}{#1}}
\newcommand{\AttributeTok}[1]{\textcolor[rgb]{0.40,0.45,0.13}{#1}}
\newcommand{\BaseNTok}[1]{\textcolor[rgb]{0.68,0.00,0.00}{#1}}
\newcommand{\BuiltInTok}[1]{\textcolor[rgb]{0.00,0.23,0.31}{#1}}
\newcommand{\CharTok}[1]{\textcolor[rgb]{0.13,0.47,0.30}{#1}}
\newcommand{\CommentTok}[1]{\textcolor[rgb]{0.37,0.37,0.37}{#1}}
\newcommand{\CommentVarTok}[1]{\textcolor[rgb]{0.37,0.37,0.37}{\textit{#1}}}
\newcommand{\ConstantTok}[1]{\textcolor[rgb]{0.56,0.35,0.01}{#1}}
\newcommand{\ControlFlowTok}[1]{\textcolor[rgb]{0.00,0.23,0.31}{#1}}
\newcommand{\DataTypeTok}[1]{\textcolor[rgb]{0.68,0.00,0.00}{#1}}
\newcommand{\DecValTok}[1]{\textcolor[rgb]{0.68,0.00,0.00}{#1}}
\newcommand{\DocumentationTok}[1]{\textcolor[rgb]{0.37,0.37,0.37}{\textit{#1}}}
\newcommand{\ErrorTok}[1]{\textcolor[rgb]{0.68,0.00,0.00}{#1}}
\newcommand{\ExtensionTok}[1]{\textcolor[rgb]{0.00,0.23,0.31}{#1}}
\newcommand{\FloatTok}[1]{\textcolor[rgb]{0.68,0.00,0.00}{#1}}
\newcommand{\FunctionTok}[1]{\textcolor[rgb]{0.28,0.35,0.67}{#1}}
\newcommand{\ImportTok}[1]{\textcolor[rgb]{0.00,0.46,0.62}{#1}}
\newcommand{\InformationTok}[1]{\textcolor[rgb]{0.37,0.37,0.37}{#1}}
\newcommand{\KeywordTok}[1]{\textcolor[rgb]{0.00,0.23,0.31}{#1}}
\newcommand{\NormalTok}[1]{\textcolor[rgb]{0.00,0.23,0.31}{#1}}
\newcommand{\OperatorTok}[1]{\textcolor[rgb]{0.37,0.37,0.37}{#1}}
\newcommand{\OtherTok}[1]{\textcolor[rgb]{0.00,0.23,0.31}{#1}}
\newcommand{\PreprocessorTok}[1]{\textcolor[rgb]{0.68,0.00,0.00}{#1}}
\newcommand{\RegionMarkerTok}[1]{\textcolor[rgb]{0.00,0.23,0.31}{#1}}
\newcommand{\SpecialCharTok}[1]{\textcolor[rgb]{0.37,0.37,0.37}{#1}}
\newcommand{\SpecialStringTok}[1]{\textcolor[rgb]{0.13,0.47,0.30}{#1}}
\newcommand{\StringTok}[1]{\textcolor[rgb]{0.13,0.47,0.30}{#1}}
\newcommand{\VariableTok}[1]{\textcolor[rgb]{0.07,0.07,0.07}{#1}}
\newcommand{\VerbatimStringTok}[1]{\textcolor[rgb]{0.13,0.47,0.30}{#1}}
\newcommand{\WarningTok}[1]{\textcolor[rgb]{0.37,0.37,0.37}{\textit{#1}}}

\providecommand{\tightlist}{%
  \setlength{\itemsep}{0pt}\setlength{\parskip}{0pt}}
\usepackage{longtable,booktabs,array}
\usepackage{calc} % for calculating minipage widths
% Correct order of tables after \paragraph or \subparagraph
\usepackage{etoolbox}
\makeatletter
\patchcmd\longtable{\par}{\if@noskipsec\mbox{}\fi\par}{}{}
\makeatother
% Allow footnotes in longtable head/foot
\IfFileExists{footnotehyper.sty}{\usepackage{footnotehyper}}{\usepackage{footnote}}
\makesavenoteenv{longtable}

\usepackage{graphicx}
\makeatletter
\def\maxwidth{\ifdim\Gin@nat@width>\linewidth\linewidth\else\Gin@nat@width\fi}
\def\maxheight{\ifdim\Gin@nat@height>\textheight\textheight\else\Gin@nat@height\fi}
\makeatother
% Scale images if necessary, so that they will not overflow the page
% margins by default, and it is still possible to overwrite the defaults
% using explicit options in \includegraphics[width, height, ...]{}
\setkeys{Gin}{width=\maxwidth,height=\maxheight,keepaspectratio}
% Set default figure placement to htbp
\makeatletter
\def\fps@figure{htbp}
\makeatother


% definitions for citeproc citations
\NewDocumentCommand\citeproctext{}{}
\NewDocumentCommand\citeproc{mm}{%
  \begingroup\def\citeproctext{#2}\cite{#1}\endgroup}
\makeatletter
 % allow citations to break across lines
 \let\@cite@ofmt\@firstofone
 % avoid brackets around text for \cite:
 \def\@biblabel#1{}
 \def\@cite#1#2{{#1\if@tempswa , #2\fi}}
\makeatother
\newlength{\cslhangindent}
\setlength{\cslhangindent}{1.5em}
\newlength{\csllabelwidth}
\setlength{\csllabelwidth}{3em}
\newenvironment{CSLReferences}[2] % #1 hanging-indent, #2 entry-spacing
 {\begin{list}{}{%
  \setlength{\itemindent}{0pt}
  \setlength{\leftmargin}{0pt}
  \setlength{\parsep}{0pt}
  % turn on hanging indent if param 1 is 1
  \ifodd #1
   \setlength{\leftmargin}{\cslhangindent}
   \setlength{\itemindent}{-1\cslhangindent}
  \fi
  % set entry spacing
  \setlength{\itemsep}{#2\baselineskip}}}
 {\end{list}}
\usepackage{calc}
\newcommand{\CSLBlock}[1]{\hfill\break\parbox[t]{\linewidth}{\strut\ignorespaces#1\strut}}
\newcommand{\CSLLeftMargin}[1]{\parbox[t]{\csllabelwidth}{\strut#1\strut}}
\newcommand{\CSLRightInline}[1]{\parbox[t]{\linewidth - \csllabelwidth}{\strut#1\strut}}
\newcommand{\CSLIndent}[1]{\hspace{\cslhangindent}#1}


\usepackage[nolongtablepatch]{lineno}
\linenumbers



\usepackage{fontspec} 

\defaultfontfeatures{Scale=MatchLowercase}
\defaultfontfeatures[\rmfamily]{Ligatures=TeX,Scale=1}

  \setmainfont[,RawFeature={fallback=mainfontfallback}]{CMU Serif}




\title{Modelling M/EEG data with Bayesian nonparametric multilevel
models}


\shorttitle{Bayesian M/EEG modelling}


\usepackage{etoolbox}









\authorsnames[{1},{2}]{Ladislas Nalborczyk,Paul Bürkner}







\authorsaffiliations{
{Aix Marseille Univ, CNRS, LPL},{TU Dortmund University, Department of
Statistics}}




\leftheader{Nalborczyk and Bürkner}



\abstract{Time-resolved electrophysoiological measurements such as those
offered by magneto- or electro-encephalography (M/EEG) provide a unique
window onto neural activity underlying cognitive process and how they
unfold over time. Typically, we are interested in testing whether such
measures differ across conditions and/or groups. The conventional
approach consists in conducting mass-univariate statistics followed by
some form of multiplicity correction (e.g., FDR, FWER) or cluster-based
inference. However, these cluster-based methods have an important
downside: they shift the focus of inference from the timepoint to the
cluster level, thus preventing any conclusion to be made about the onset
and offset of effects (e.g., differences across conditions). Here, we
introduce a novel \emph{model-based approch} for analysing
one-dimensional M/EEG timeseries such as ERPs or decoding timecourses
and their differences across conditions. This approach relies on
Bayesian nonparametric multilevel modelling (multilevel generalised
additive models or Gaussian processes), which outputs posterior
probabililities of the effect being above chance at every
timestep/voxel, while naturally taking into account the temporal
dependencies and between-subject variability present in such data.}

\keywords{EEG, MEG, generalised additive models, gaussian
processes, mixed-effects models, Bayesian statistics, brms}

\authornote{\par{\addORCIDlink{Ladislas
Nalborczyk}{0000-0002-7419-9855}}\par{\addORCIDlink{Paul
Bürkner}{0000-0001-5765-8995}} 

\par{   The authors have no conflicts of interest to disclose.    }
\par{Correspondence concerning this article should be addressed
to Ladislas Nalborczyk, Aix Marseille Univ, CNRS, LPL, 5 avenue
Pasteur, 13100
Aix-en-Provence, France, email: ladislas.nalborczyk@cnrs.fr}
}

\makeatletter
\let\endoldlt\endlongtable
\def\endlongtable{
\hline
\endoldlt
}
\makeatother

\urlstyle{same}



\usepackage{mathtools}
\makeatletter
\@ifpackageloaded{caption}{}{\usepackage{caption}}
\AtBeginDocument{%
\ifdefined\contentsname
  \renewcommand*\contentsname{Table of contents}
\else
  \newcommand\contentsname{Table of contents}
\fi
\ifdefined\listfigurename
  \renewcommand*\listfigurename{List of Figures}
\else
  \newcommand\listfigurename{List of Figures}
\fi
\ifdefined\listtablename
  \renewcommand*\listtablename{List of Tables}
\else
  \newcommand\listtablename{List of Tables}
\fi
\ifdefined\figurename
  \renewcommand*\figurename{Figure}
\else
  \newcommand\figurename{Figure}
\fi
\ifdefined\tablename
  \renewcommand*\tablename{Table}
\else
  \newcommand\tablename{Table}
\fi
}
\@ifpackageloaded{float}{}{\usepackage{float}}
\floatstyle{ruled}
\@ifundefined{c@chapter}{\newfloat{codelisting}{h}{lop}}{\newfloat{codelisting}{h}{lop}[chapter]}
\floatname{codelisting}{Listing}
\newcommand*\listoflistings{\listof{codelisting}{List of Listings}}
\makeatother
\makeatletter
\makeatother
\makeatletter
\@ifpackageloaded{caption}{}{\usepackage{caption}}
\@ifpackageloaded{subcaption}{}{\usepackage{subcaption}}
\makeatother

% From https://tex.stackexchange.com/a/645996/211326
%%% apa7 doesn't want to add appendix section titles in the toc
%%% let's make it do it
\makeatletter
\xpatchcmd{\appendix}
  {\par}
  {\addcontentsline{toc}{section}{\@currentlabelname}\par}
  {}{}
\makeatother

%% Disable longtable counter
%% https://tex.stackexchange.com/a/248395/211326

\usepackage{etoolbox}

\makeatletter
\patchcmd{\LT@caption}
  {\bgroup}
  {\bgroup\global\LTpatch@captiontrue}
  {}{}
\patchcmd{\longtable}
  {\par}
  {\par\global\LTpatch@captionfalse}
  {}{}
\apptocmd{\endlongtable}
  {\ifLTpatch@caption\else\addtocounter{table}{-1}\fi}
  {}{}
\newif\ifLTpatch@caption
\makeatother

\begin{document}

\maketitle

\hypertarget{toc}{}
\tableofcontents
\newpage
\section[Introduction]{Modelling M/EEG data with Bayesian nonparametric
multilevel models}

\setcounter{secnumdepth}{-\maxdimen} % remove section numbering

\setlength\LTleft{0pt}

\resetlinenumber[1]

\section{Introduction (in progress)}\label{introduction-in-progress}

Here are some useful references to be discussed
(\citeproc{ref-combrisson_exceeding_2015}{Combrisson \& Jerbi, 2015};
\citeproc{ref-ehinger_unfold_2019}{Ehinger \& Dimigen, 2019};
\citeproc{ref-frossard2021}{Frossard \& Renaud, 2021a},
\citeproc{ref-frossard2022}{2022}; \citeproc{ref-gramfort2013}{Gramfort,
2013}; \citeproc{ref-hayasaka_validating_2003}{Hayasaka, 2003};
\citeproc{ref-luck_how_2017}{Luck \& Gaspelin, 2017};
\citeproc{ref-maris2007}{Maris \& Oostenveld, 2007};
\citeproc{ref-pedersen_hierarchical_2019}{E. J. Pedersen et al., 2019};
\citeproc{ref-pernet2015}{Pernet et al., 2015};
\citeproc{ref-riutort-mayol_practical_2023}{Riutort-Mayol et al., 2023};
\citeproc{ref-rousselet_using_2025}{Rousselet, 2025})\ldots{} See also
(\citeproc{ref-maris2011}{Maris, 2011})\ldots{} and
(\citeproc{ref-rosenblatt2018}{Rosenblatt et al., 2018}) (history of
cluster-based approaches and using a data split?)\ldots{} Cluster
failure (\citeproc{ref-eklund2016}{Eklund et al., 2016})\ldots{}

In the following, we consider two approaches to modelling the non-linear
timecourse of M/EEG ERPs or decoding performance: i) generalised
additive models (GAMs) with thin-plate smoothing splines
(\citeproc{ref-wood2003}{Wood, 2003}; \citeproc{ref-wood2004}{Wood,
2004}) and ii) Gaussian processes (GPs) with a smooth covariance kernel
(\citeproc{ref-rasmussen2005}{Rasmussen \& Williams, 2005}) and low-rank
approximation (\citeproc{ref-riutort-mayol_practical_2023}{Riutort-Mayol
et al., 2023}).

\subsection{Previous modelling work}\label{previous-modelling-work}

Disentangling overlapping processes
(\citeproc{ref-skukies_brain_2024}{Skukies et al., 2024};
\citeproc{ref-skukies_modelling_2021}{Skukies \& Ehinger, 2021})\ldots{}
Using Bayes factors (\citeproc{ref-teichmann2022}{Teichmann,
2022})\ldots{}

\subsection{Generalised additive
models}\label{generalised-additive-models}

In generalised additive models (GAMs), the functional relationship
between predictors and response variable is decomposed into a sum of
low-dimensional non-parametric functions. A typical GAM has the
following form:

\[
\begin{aligned} 
y_{i} &\sim \mathrm{EF}\left(\mu_{i}, \phi\right)\\
g\left(\mu_i\right) &= A_{i} + \mathbf{X}_{i} \gamma + \sum_{j=1}^{J} f_{j}\left(x_{ij}\right)
\end{aligned}
\]

where \(y_{i} \sim \mathrm{EF}\left(\mu_{i}, \phi\right)\) denotes that
the observations \(y_{i}\) are distributed as some member of the
exponential family of distributions (e.g., Gaussian, Gamma, Beta,
Poisson) with mean \(\mu_{i}\) and scale parameter \(\phi\);
\(g(\cdot)\) is the link function, \(A\) is an offset,
\(\mathbf{X}_{i}\) is the \(i\)th row of a parametric model matrix,
\(\gamma\) is a vector of parameters for the parametric terms, \(f_{j}\)
is a smooth function of covariate \(x_{j}\). The smooth functions
\(f_{j}\) are represented in the model via penalised splines basis
expansions of the covariates, that are a weighted sum of basis
functions:

\[
f_{j}\left(x_{i j}\right) = \sum_{k=1}^K \beta_{jk} b_{jk}\left(x_{ij}\right)
\]

where \(\beta_{jk}\) is the weight (coefficient) associated with the
\(k\)th basis function \(b_{jk}()\) evaluated at the covariate value
\(x_{ij}\) for the \(j\)th smooth function \(f_{j}\). Splines'
coefficients are penalised\ldots{}

\subsection{Gaussian process
regression}\label{gaussian-process-regression}

A Gaussian process (GP) is a stochastic process that defines the
distribution over a collection of random variables indexed by a
continuous variable, that is \(\{f(t): t \in \mathcal{T}\}\) for some
index set \(\mathcal{T}\) (\citeproc{ref-rasmussen2005}{Rasmussen \&
Williams, 2005};
\citeproc{ref-riutort-mayol_practical_2023}{Riutort-Mayol et al.,
2023}). Whereas Bayesian linear regression outputs a distribution over
the parameters of some predefind parametric model, the GP approach, in
contrast, is a non-parametric approach, in that it finds a distribution
over the possible functions that are consistent with the observed data.
However, note that nonparametric does not mean there aren't parameters,
it means that there are infinitely many parameters.

From
\href{https://www.rdocumentation.org/packages/brms/versions/2.22.0/topics/gp}{brms
documentation}: A GP is a stochastic process, which describes the
relation between one or more predictors
\(x=\left(x_{1}, \ldots, x_{d}\right)\) and a response \(f(x)\), where
\(d\) is the number of predictors. A GP is the generalization of the
multivariate normal distribution to an infinite number of dimensions.
Thus, it can be interpreted as a prior over functions. The values of
\(f()\) at any finite set of locations are jointly multivariate normal,
with a covariance matrix defined by the covariance kernel
\(k_p\left(x_i, x_j\right)\), where \(p\) is the vector of parameters of
the GP:

\[
\left(f\left(x_{1}\right), \ldots f\left(x_{n}\right) \sim \operatorname{MVN}\left(0, \left(k_p\left(x_{i}, x_{j}\right)\right)_{i, j=1}^{n}\right)\right.
\]

The smoothness and general behaviour of the function \(f\) depends only
on the choice of covariance kernel, which ensures that values that are
close together in the input space will be mapped to similar output
values\ldots{}

From this perspective, \(f\) is a realisation of an infinite dimensional
normal distribution:

\[
f \sim \mathrm{Normal}(0, C(\lambda))
\]

where \(C\) is a covariance kernel with hyperparameters \(\lambda\) that
defines the covariance between two function values \(f\left(t_1\right)\)
and \(f\left(t_2\right)\) for two time points \(t_1\) and \(t_2\)
(\citeproc{ref-rasmussen2005}{Rasmussen \& Williams, 2005}). Similar to
the different choices of the basis function for splines, different
choices of the covariance kernel lead to different GPs. In this article,
we consider the squared-exponential (a.k.a. radial basis function)
kernel, which computes the squared distance between points and converts
it into a measure of similarity. It is defined as:

\[
C(\lambda) := C\left(t_1, t_2, \sigma, \gamma\right) := \sigma^2 \exp \left(-\frac{||t_1-t_2||^{2}}{2 \gamma^2}\right)
\]

with hyperparameters \(\lambda = (\sigma, \gamma)\), expressing the
overall scale of GP and the length-scale, respectively
(\citeproc{ref-rasmussen2005}{Rasmussen \& Williams, 2005}). The
advantages of this kernel are that it is computationally efficient and
(infinitely) smooth making it a reasonable choice for the purposes of
the present article. Here again, \(\lambda\) hyperparameters are
estimated from the data, along with all other model parameters.

Taken from
\url{https://michael-franke.github.io/Bayesian-Regression/practice-sheets/10c-Gaussian-processes.html}:
For a given vector \(\mathbf{x}\), we can use the kernel to construct
finite multi-variate normal distribution associated with it like so:

\[
\mathbf{x} \mapsto_{G P} \operatorname{MVNormal}(m(\mathbf{x}), k(\mathbf{x}, \mathbf{x}))
\]

where \(m\) is a function that specifies the mean for the distribution
associated with \(\mathbf{x}\). This mapping is essentially the Gaussian
process: a systematic association of vectors of arbitrary length with a
suitable multi-variate normal distribution.

Low-rank approximate Gaussian processes are of main interest in machine
learning and statistics due to the high computational demands of exact
Gaussian process models
(\citeproc{ref-riutort-mayol_practical_2023}{Riutort-Mayol et al.,
2023})\ldots{}

\subsection{Objectives}\label{objectives}

\ldots{}

\section{Methods}\label{methods}

\subsection{M/EEG data simulation}\label{meeg-data-simulation}

Following the approach used by Sassenhagen \& Draschkow
(\citeproc{ref-sassenhagen2019}{2019}) and Rousselet
(\citeproc{ref-rousselet_using_2025}{2025}), we simulated EEG data
stemming from two conditions, one with noise only, and the other with
noise + signal. As in previous studies, the noise was generated by
superimposing 50 sinusoids at different frequencies, following an
EEG-like spectrum (see details and code in
\citeproc{ref-yeung2004}{Yeung et al., 2004}). As in Rousselet
(\citeproc{ref-rousselet_using_2025}{2025}), the signal was generated
from truncated Gaussian with an objective onset at 160 ms, a peak at 250
ms, and an offset at 342 ms. We simulated this signal for 250 timesteps
between 0 and 0.5s, akin to \& 500 Hz sampling rate. We simulated such
data for a group of 20 participants with 50 trials per participant and
condition.

\begin{figure}[H]

\caption{Some ERPs in two conditions with 50 trials each, for a group of
20 participants.}

{\centering \includegraphics[width=1\textwidth,height=\textheight]{brms_meeg_files/figure-pdf/eeg-group-1.pdf}

}

\end{figure}%

\begin{Shaded}
\begin{Highlighting}[]
\CommentTok{\# averaging across participants}
\NormalTok{group\_df }\OtherTok{\textless{}{-}}\NormalTok{ raw\_df }\SpecialCharTok{\%\textgreater{}\%}
    \FunctionTok{pivot\_wider}\NormalTok{(}\AttributeTok{names\_from =}\NormalTok{ condition, }\AttributeTok{values\_from =}\NormalTok{ y, }\AttributeTok{values\_fn =}\NormalTok{ mean) }\SpecialCharTok{\%\textgreater{}\%}
    \FunctionTok{mutate}\NormalTok{(}\AttributeTok{y\_diff =}\NormalTok{ cond2 }\SpecialCharTok{{-}}\NormalTok{ cond1) }\SpecialCharTok{\%\textgreater{}\%}
    \FunctionTok{summarise}\NormalTok{(}
        \AttributeTok{y\_mean =} \FunctionTok{mean}\NormalTok{(y\_diff),}
        \AttributeTok{y\_sem =} \FunctionTok{sd}\NormalTok{(y\_diff) }\SpecialCharTok{/} \FunctionTok{sqrt}\NormalTok{(}\FunctionTok{n}\NormalTok{() ),}
        \AttributeTok{.by =}\NormalTok{ x}
\NormalTok{        ) }\SpecialCharTok{\%\textgreater{}\%}
    \FunctionTok{mutate}\NormalTok{(}\AttributeTok{lower =}\NormalTok{ y\_mean }\SpecialCharTok{{-}}\NormalTok{ y\_sem, }\AttributeTok{upper =}\NormalTok{ y\_mean }\SpecialCharTok{+}\NormalTok{ y\_sem)}

\CommentTok{\# plotting the data}
\NormalTok{group\_df }\SpecialCharTok{\%\textgreater{}\%}
    \FunctionTok{ggplot}\NormalTok{(}\FunctionTok{aes}\NormalTok{(}\AttributeTok{x =}\NormalTok{ x, }\AttributeTok{y =}\NormalTok{ y\_mean) ) }\SpecialCharTok{+}
    \FunctionTok{geom\_hline}\NormalTok{(}\AttributeTok{yintercept =} \DecValTok{0}\NormalTok{, }\AttributeTok{linetype =} \DecValTok{2}\NormalTok{) }\SpecialCharTok{+}
    \FunctionTok{geom\_vline}\NormalTok{(}\AttributeTok{xintercept =}\NormalTok{ true\_onset, }\AttributeTok{linetype =} \DecValTok{2}\NormalTok{) }\SpecialCharTok{+}
    \FunctionTok{geom\_ribbon}\NormalTok{(}
        \FunctionTok{aes}\NormalTok{(}\AttributeTok{ymin =}\NormalTok{ lower, }\AttributeTok{ymax =}\NormalTok{ upper, }\AttributeTok{colour =} \ConstantTok{NULL}\NormalTok{),}
        \AttributeTok{alpha =} \FloatTok{0.25}\NormalTok{, }\AttributeTok{show.legend =} \ConstantTok{FALSE}
\NormalTok{        ) }\SpecialCharTok{+}
    \FunctionTok{geom\_line}\NormalTok{(}\AttributeTok{show.legend =} \ConstantTok{FALSE}\NormalTok{) }\SpecialCharTok{+}
    \FunctionTok{labs}\NormalTok{(}\AttributeTok{x =} \StringTok{"Time (ms)"}\NormalTok{, }\AttributeTok{y =} \StringTok{"EEG difference (a.u.)"}\NormalTok{)}
\end{Highlighting}
\end{Shaded}

\begin{figure}[H]

\caption{Group-level average difference between conditions (mean +/-
standard error of the mean). The `true' (population value of the) onset
is indicated by the vertical dashed line.}

{\centering \includegraphics[width=0.75\textwidth,height=\textheight]{brms_meeg_files/figure-pdf/erp-diff-1.pdf}

}

\end{figure}%

\subsection{Model fitting}\label{model-fitting}

Models were fitted using the \texttt{brms} package
(\citeproc{ref-brms2017}{Bürkner, 2017}, \citeproc{ref-brms2018}{2018};
\citeproc{ref-nalborczyk2019}{Nalborczyk et al., 2019})\ldots{}

\begin{Shaded}
\begin{Highlighting}[]
\CommentTok{\# defining a contrast for condition}
\FunctionTok{contrasts}\NormalTok{(group\_df}\SpecialCharTok{$}\NormalTok{condition) }\OtherTok{\textless{}{-}} \FunctionTok{c}\NormalTok{(}\SpecialCharTok{{-}}\FloatTok{0.5}\NormalTok{, }\FloatTok{0.5}\NormalTok{)}

\CommentTok{\# fitting the GAM}
\NormalTok{gam }\OtherTok{\textless{}{-}} \FunctionTok{brm}\NormalTok{(}
    \CommentTok{\# cubic regression splines with k{-}1 basis functions}
    \CommentTok{\# then we should try s(x, participant, bs = "fs")}
    \CommentTok{\# that is, random factor smooth interactions...}
\NormalTok{    y }\SpecialCharTok{\textasciitilde{}}\NormalTok{ condition }\SpecialCharTok{+} \FunctionTok{s}\NormalTok{(x, }\AttributeTok{bs =} \StringTok{"cr"}\NormalTok{, }\AttributeTok{k =} \DecValTok{20}\NormalTok{, }\AttributeTok{by =}\NormalTok{ condition),}
    \AttributeTok{data =}\NormalTok{ group\_df,}
    \AttributeTok{family =} \FunctionTok{gaussian}\NormalTok{(),}
    \AttributeTok{iter =} \DecValTok{5000}\NormalTok{,}
    \AttributeTok{chains =} \DecValTok{4}\NormalTok{,}
    \AttributeTok{cores =} \DecValTok{4}\NormalTok{,}
    \AttributeTok{file =} \StringTok{"models/gam.rds"}
\NormalTok{    )}
\end{Highlighting}
\end{Shaded}

Now we fit the GP\ldots{}

\begin{Shaded}
\begin{Highlighting}[]
\NormalTok{gp\_model }\OtherTok{\textless{}{-}} \FunctionTok{brm}\NormalTok{(}
    \CommentTok{\# k refers to the number of basis functions for}
    \CommentTok{\# computing Hilbert{-}space approximate GPs}
\NormalTok{    y }\SpecialCharTok{\textasciitilde{}} \FunctionTok{gp}\NormalTok{(x, }\AttributeTok{k =} \DecValTok{20}\NormalTok{, }\AttributeTok{by =}\NormalTok{ condition),}
    \CommentTok{\# if k = NA (default), exact GPs are computed}
    \CommentTok{\# y \textasciitilde{} gp(x, by = condition),}
    \AttributeTok{data =}\NormalTok{ group\_df,}
    \AttributeTok{family =} \FunctionTok{gaussian}\NormalTok{(),}
    \AttributeTok{control =} \FunctionTok{list}\NormalTok{(}\AttributeTok{adapt\_delta =} \FloatTok{0.99}\NormalTok{),}
    \AttributeTok{iter =} \DecValTok{5000}\NormalTok{,}
    \AttributeTok{chains =} \DecValTok{4}\NormalTok{,}
    \AttributeTok{cores =} \DecValTok{4}\NormalTok{,}
    \AttributeTok{file =} \StringTok{"models/gp.rds"}
\NormalTok{    )}
\end{Highlighting}
\end{Shaded}

And we plot the posterior predictions\ldots{}

\begin{Shaded}
\begin{Highlighting}[]
\CommentTok{\# plotting the posterior predictions}
\FunctionTok{plot\_post\_preds}\NormalTok{(}\AttributeTok{model =}\NormalTok{ gam, }\AttributeTok{data =}\NormalTok{ group\_df) }\SpecialCharTok{+}
    \FunctionTok{plot\_post\_preds}\NormalTok{(}\AttributeTok{model =}\NormalTok{ gp\_model, }\AttributeTok{data =}\NormalTok{ group\_df)}
\end{Highlighting}
\end{Shaded}

\begin{figure}[H]

\caption{Posterior predictions of the GAM (left) and GP (right) models.}

{\centering \includegraphics[width=1\textwidth,height=\textheight]{brms_meeg_files/figure-pdf/gam-gp-preds-1.pdf}

}

\end{figure}%

\subsection{Posterior probability of difference above
0}\label{posterior-probability-of-difference-above-0}

We can retrieve the posterior probability of the slope for
\texttt{condition} at each timestep.

\begin{figure}[H]

\caption{Slope for the difference between conditions according to the
GAM (left) or the GP (right).}

{\centering \includegraphics[width=1\textwidth,height=\textheight]{brms_meeg_files/figure-pdf/unnamed-chunk-1-1.pdf}

}

\end{figure}%

We can also compute the posterior probability of the slope for
\texttt{condition} being above 0 (Figure~\ref{fig-post-prob-test}).

\begin{figure}[!htb]

\caption{\label{fig-post-prob-test}Posterior probability of the EEG
signal being above 0+eps according to the GAM (left) and the GP
(right).}

\centering{

\includegraphics[width=1\textwidth,height=\textheight]{brms_meeg_files/figure-pdf/fig-post-prob-test-1.pdf}

}

\end{figure}%

We can also express this as the ratio of posterior probabilities (i.e.,
\(p/(1-p)\)) and visualise the timecourse of this ratio superimposed
with the conventional BF thresholds (Figure~\ref{fig-post-prob-ratio}).

\begin{figure}[!htb]

\caption{\label{fig-post-prob-ratio}Ratio of posterior probability
according to the GAM (left) and the GP (right). Timesteps above
threshold (\textgreater10) are highlighted in green.}

\centering{

\includegraphics[width=1\textwidth,height=\textheight]{brms_meeg_files/figure-pdf/fig-post-prob-ratio-1.pdf}

}

\end{figure}%

\newpage

\subsection{Error properties of the proposed
approach}\label{error-properties-of-the-proposed-approach}

Below we are plotting the error function (where the error is computed as
\(|\hat{\theta}-\theta|\)) for both the onset and offset values of the
ERP difference, according to various \texttt{eps} and \texttt{threshold}
values. Remember that the signal is generated from a truncated Gaussian
defining an objective onset at 160 ms, a maximum at 250 ms, and an
offset at 342 ms. Figure~\ref{fig-onset-error} shows that the GP model
can \emph{exactly} recover the true onset and offset values, given some
reasonable choice of \texttt{eps} and \texttt{threshold} values.

\begin{verbatim}
   eps threshold estimated_onset estimated_offset error_onset error_offset
1 0.06        15             160              342           0            0
2 0.06        16             160              342           0            0
3 0.06        17             160              342           0            0
4 0.06        18             160              342           0            0
5 0.06        19             160              342           0            0
6 0.06        20             160              342           0            0
\end{verbatim}

\begin{verbatim}
   eps threshold estimated_onset estimated_offset error_onset error_offset
1 0.06         3             159              342           1            0
2 0.06         4             159              342           1            0
3 0.06         5             159              342           1            0
4 0.06         6             159              342           1            0
5 0.06         7             159              342           1            0
6 0.06         8             159              342           1            0
\end{verbatim}

\begin{figure}[!htb]

\caption{\label{fig-onset-error}Error function of onset (left) and
offset (right) estimation according to various eps and threshold values
(according to the GP model). Minimum error values are indicated by red
crosses.}

\centering{

\includegraphics[width=1\textwidth,height=\textheight]{brms_meeg_files/figure-pdf/fig-onset-error-1.pdf}

}

\end{figure}%

\newpage

\subsection{Using summary statistics of
ERPs}\label{using-summary-statistics-of-erps}

Next we fit a hierarchical GAM using summary statistics of ERPs (mean
and SD) at the participant level (similar to what is done in
meta-analysis).

\begin{Shaded}
\begin{Highlighting}[]
\CommentTok{\# averaging across participants}
\NormalTok{summary\_df }\OtherTok{\textless{}{-}}\NormalTok{ raw\_df }\SpecialCharTok{\%\textgreater{}\%}
    \FunctionTok{summarise}\NormalTok{(}
        \AttributeTok{eeg =} \FunctionTok{mean}\NormalTok{(y),}
        \AttributeTok{eeg\_sd =} \FunctionTok{sd}\NormalTok{(y),}
        \AttributeTok{.by =} \FunctionTok{c}\NormalTok{(participant, condition, x)}
\NormalTok{        )}

\CommentTok{\# defining a contrast for condition}
\FunctionTok{contrasts}\NormalTok{(summary\_df}\SpecialCharTok{$}\NormalTok{condition) }\OtherTok{\textless{}{-}} \FunctionTok{c}\NormalTok{(}\SpecialCharTok{{-}}\FloatTok{0.5}\NormalTok{, }\FloatTok{0.5}\NormalTok{)}

\CommentTok{\# fitting the GAM}
\NormalTok{meta\_gam }\OtherTok{\textless{}{-}} \FunctionTok{brm}\NormalTok{(}
    \CommentTok{\# using by{-}participant SD of ERPs across trials}
\NormalTok{    eeg }\SpecialCharTok{|} \FunctionTok{se}\NormalTok{(eeg\_sd) }\SpecialCharTok{\textasciitilde{}}
\NormalTok{        condition }\SpecialCharTok{+} \FunctionTok{s}\NormalTok{(x, }\AttributeTok{bs =} \StringTok{"cr"}\NormalTok{, }\AttributeTok{k =} \DecValTok{10}\NormalTok{, }\AttributeTok{by =}\NormalTok{ condition) }\SpecialCharTok{+}
\NormalTok{        (}\DecValTok{1} \SpecialCharTok{|}\NormalTok{ participant),}
    \AttributeTok{data =}\NormalTok{ summary\_df,}
    \AttributeTok{family =} \FunctionTok{gaussian}\NormalTok{(),}
    \AttributeTok{iter =} \DecValTok{5000}\NormalTok{,}
    \AttributeTok{chains =} \DecValTok{4}\NormalTok{,}
    \AttributeTok{cores =} \DecValTok{4}\NormalTok{,}
    \AttributeTok{file =} \StringTok{"models/meta\_gam.rds"}
\NormalTok{    )}
\end{Highlighting}
\end{Shaded}

\begin{Shaded}
\begin{Highlighting}[]
\CommentTok{\# plotting the posterior predictions}
\FunctionTok{plot}\NormalTok{(}
    \FunctionTok{conditional\_effects}\NormalTok{(}\AttributeTok{x =}\NormalTok{ meta\_gam, }\AttributeTok{effect =} \StringTok{"x:condition"}\NormalTok{),}
    \AttributeTok{points =} \ConstantTok{FALSE}\NormalTok{, }\AttributeTok{theme =} \FunctionTok{theme\_light}\NormalTok{(), }\AttributeTok{plot =} \ConstantTok{FALSE}
\NormalTok{    )[[}\DecValTok{1}\NormalTok{]] }\SpecialCharTok{+}
    \FunctionTok{labs}\NormalTok{(}\AttributeTok{x =} \StringTok{"Time (ms)"}\NormalTok{, }\AttributeTok{y =} \StringTok{"EEG signal (a.u.)"}\NormalTok{)}
\end{Highlighting}
\end{Shaded}

\begin{figure}[H]

\caption{Hierarchical GAM posterior predictions.}

{\centering \includegraphics[width=1\textwidth,height=\textheight]{brms_meeg_files/figure-pdf/meta-gam-preds-1.pdf}

}

\end{figure}%

\begin{Shaded}
\begin{Highlighting}[]
\CommentTok{\# fitting the GP}
\NormalTok{meta\_gp }\OtherTok{\textless{}{-}} \FunctionTok{brm}\NormalTok{(}
    \CommentTok{\# using by{-}participant SD of ERPs across trials}
\NormalTok{    eeg }\SpecialCharTok{|} \FunctionTok{se}\NormalTok{(eeg\_sd) }\SpecialCharTok{\textasciitilde{}}
\NormalTok{        condition }\SpecialCharTok{+} \FunctionTok{gp}\NormalTok{(x, }\AttributeTok{k =} \DecValTok{20}\NormalTok{, }\AttributeTok{by =}\NormalTok{ condition) }\SpecialCharTok{+}
\NormalTok{        (}\DecValTok{1} \SpecialCharTok{|}\NormalTok{ participant),}
    \AttributeTok{data =}\NormalTok{ summary\_df,}
    \AttributeTok{family =} \FunctionTok{gaussian}\NormalTok{(),}
    \AttributeTok{control =} \FunctionTok{list}\NormalTok{(}\AttributeTok{adapt\_delta =} \FloatTok{0.95}\NormalTok{, }\AttributeTok{max\_treedepth =} \DecValTok{20}\NormalTok{),}
    \AttributeTok{iter =} \DecValTok{5000}\NormalTok{,}
    \AttributeTok{chains =} \DecValTok{4}\NormalTok{,}
    \AttributeTok{cores =} \DecValTok{4}\NormalTok{,}
    \AttributeTok{file =} \StringTok{"models/meta\_gp.rds"}
\NormalTok{    )}
\end{Highlighting}
\end{Shaded}

\begin{Shaded}
\begin{Highlighting}[]
\CommentTok{\# plotting the posterior predictions}
\FunctionTok{plot}\NormalTok{(}
    \FunctionTok{conditional\_effects}\NormalTok{(}\AttributeTok{x =}\NormalTok{ meta\_gp, }\AttributeTok{effect =} \StringTok{"x:condition"}\NormalTok{),}
    \AttributeTok{points =} \ConstantTok{FALSE}\NormalTok{, }\AttributeTok{theme =} \FunctionTok{theme\_light}\NormalTok{(), }\AttributeTok{plot =} \ConstantTok{FALSE}
\NormalTok{    )[[}\DecValTok{1}\NormalTok{]] }\SpecialCharTok{+}
    \FunctionTok{labs}\NormalTok{(}\AttributeTok{x =} \StringTok{"Time (ms)"}\NormalTok{, }\AttributeTok{y =} \StringTok{"EEG signal (a.u.)"}\NormalTok{)}
\end{Highlighting}
\end{Shaded}

\subsection{Application to 1D decoding results (accuracy over
time)}\label{application-to-1d-decoding-results-accuracy-over-time}

Assume we have M/EEG data and we conducted time-resolved multivariate
pattern analysis (MVPA), also known as decoding. As a result, we have a
timecourse of decoding accuracies (e.g., ROC AUC), bounded between 0 and
1, per participant (Figure~\ref{fig-sim-decoding}).

\begin{figure}[!htb]

\caption{\label{fig-sim-decoding}Exemplary (simulated) group-level
average timecourse of binary decoding accuracy (ROC AUC).}

\centering{

\includegraphics[width=0.75\textwidth,height=\textheight]{brms_meeg_files/figure-pdf/fig-sim-decoding-1.pdf}

}

\end{figure}%

\begin{Shaded}
\begin{Highlighting}[]
\CommentTok{\# loading the reticulate package}
\FunctionTok{library}\NormalTok{(reticulate)}

\CommentTok{\# importing the numpy and mne python modules}
\NormalTok{np }\OtherTok{\textless{}{-}} \FunctionTok{import}\NormalTok{(}\StringTok{"numpy"}\NormalTok{)}
\NormalTok{mne }\OtherTok{\textless{}{-}} \FunctionTok{import}\NormalTok{(}\StringTok{"mne"}\NormalTok{)}
\NormalTok{sklearn }\OtherTok{\textless{}{-}} \FunctionTok{import}\NormalTok{(}\StringTok{"sklearn"}\NormalTok{)}

\CommentTok{\# defining the function in R (it will be executed in Python)}
\NormalTok{mne\_decoding }\OtherTok{\textless{}{-}} \ControlFlowTok{function}\NormalTok{ (X, labels, }\AttributeTok{ncores =} \DecValTok{1}\NormalTok{) \{}
    
    \CommentTok{\# converting R dataframe to NumPy array (reshaping if needed)}
    \CommentTok{\# X should be a matrix before conversion}
\NormalTok{    X\_np }\OtherTok{\textless{}{-}}\NormalTok{ np}\SpecialCharTok{$}\FunctionTok{array}\NormalTok{(X)}
    
    \ControlFlowTok{if}\NormalTok{ (}\FunctionTok{length}\NormalTok{(}\FunctionTok{dim}\NormalTok{(X\_np) ) }\SpecialCharTok{==} \DecValTok{2}\NormalTok{) \{}
        
        \CommentTok{\# adding a second dimension (channels) if missing}
\NormalTok{        X\_np }\OtherTok{\textless{}{-}}\NormalTok{ np}\SpecialCharTok{$}\FunctionTok{expand\_dims}\NormalTok{(X\_np, }\AttributeTok{axis =} \FunctionTok{as.integer}\NormalTok{(}\DecValTok{1}\NormalTok{) )}
        
\NormalTok{    \}}
  
    \CommentTok{\# defining the classifier}
\NormalTok{    clf }\OtherTok{\textless{}{-}}\NormalTok{ sklearn}\SpecialCharTok{$}\NormalTok{linear\_model}\SpecialCharTok{$}\FunctionTok{LogisticRegression}\NormalTok{(}\AttributeTok{solver =} \StringTok{"liblinear"}\NormalTok{)}
    
    \CommentTok{\# sliding the estimator on all time frames}
\NormalTok{    time\_decod }\OtherTok{\textless{}{-}}\NormalTok{ mne}\SpecialCharTok{$}\NormalTok{decoding}\SpecialCharTok{$}\FunctionTok{SlidingEstimator}\NormalTok{(}
\NormalTok{        clf, }\AttributeTok{n\_jobs =} \FunctionTok{as.integer}\NormalTok{(ncores),}
        \AttributeTok{scoring =} \StringTok{"roc\_auc"}\NormalTok{, }\AttributeTok{verbose =} \ConstantTok{TRUE}
\NormalTok{        )}
    
    \CommentTok{\# or using N{-}fold cross{-}validation}
\NormalTok{    scores }\OtherTok{\textless{}{-}}\NormalTok{ mne}\SpecialCharTok{$}\NormalTok{decoding}\SpecialCharTok{$}\FunctionTok{cross\_val\_multiscore}\NormalTok{(}
\NormalTok{        time\_decod,}
\NormalTok{        X\_np,}
\NormalTok{        labels,}
        \AttributeTok{cv =} \FunctionTok{as.integer}\NormalTok{(}\DecValTok{4}\NormalTok{),}
        \AttributeTok{n\_jobs =} \FunctionTok{as.integer}\NormalTok{(ncores),}
        \AttributeTok{verbose =} \ConstantTok{TRUE}
\NormalTok{        )}
    
    \CommentTok{\# returning the scores (averaged over CV folds)}
    \FunctionTok{return}\NormalTok{ (scores)}
    
\NormalTok{\}}

\CommentTok{\# listing all participants}
\NormalTok{participants }\OtherTok{\textless{}{-}} \FunctionTok{unique}\NormalTok{(raw\_df}\SpecialCharTok{$}\NormalTok{participant)}

\CommentTok{\# initialising empty decoding results}
\NormalTok{group\_decoding\_scores }\OtherTok{\textless{}{-}} \FunctionTok{data.frame}\NormalTok{()}

\CommentTok{\# running decoding for each participant}
\ControlFlowTok{for}\NormalTok{ (ppt }\ControlFlowTok{in}\NormalTok{ participants) \{}
    
    \CommentTok{\# printing progress}
    \FunctionTok{print}\NormalTok{(ppt)}
    
    \CommentTok{\# retrieve data from one participant}
\NormalTok{    ppt\_data }\OtherTok{\textless{}{-}}\NormalTok{ raw\_df }\SpecialCharTok{\%\textgreater{}\%}
        \FunctionTok{filter}\NormalTok{(participant }\SpecialCharTok{==}\NormalTok{ ppt) }\SpecialCharTok{\%\textgreater{}\%}
        \FunctionTok{select}\NormalTok{(}\SpecialCharTok{{-}}\NormalTok{participant) }\SpecialCharTok{\%\textgreater{}\%}
        \FunctionTok{pivot\_wider}\NormalTok{(}\AttributeTok{names\_from =}\NormalTok{ x, }\AttributeTok{values\_from =}\NormalTok{ y) }\SpecialCharTok{\%\textgreater{}\%}
        \FunctionTok{select}\NormalTok{(}\SpecialCharTok{{-}}\NormalTok{condition, }\SpecialCharTok{{-}}\NormalTok{trial)}
        
    \CommentTok{\# extracting the labels}
\NormalTok{    labels }\OtherTok{\textless{}{-}}\NormalTok{ raw\_df }\SpecialCharTok{\%\textgreater{}\%}
        \FunctionTok{filter}\NormalTok{(participant }\SpecialCharTok{==}\NormalTok{ ppt) }\SpecialCharTok{\%\textgreater{}\%}
        \FunctionTok{select}\NormalTok{(}\SpecialCharTok{{-}}\NormalTok{participant) }\SpecialCharTok{\%\textgreater{}\%}
        \FunctionTok{pivot\_wider}\NormalTok{(}\AttributeTok{names\_from =}\NormalTok{ x, }\AttributeTok{values\_from =}\NormalTok{ y) }\SpecialCharTok{\%\textgreater{}\%}
        \FunctionTok{pull}\NormalTok{(condition) }\SpecialCharTok{\%\textgreater{}\%}
        \FunctionTok{as.numeric}\NormalTok{()}
    
    \CommentTok{\# extracting the timesteps}
\NormalTok{    timesteps }\OtherTok{\textless{}{-}}\NormalTok{ raw\_df }\SpecialCharTok{\%\textgreater{}\%}
        \FunctionTok{filter}\NormalTok{(participant }\SpecialCharTok{==}\NormalTok{ ppt) }\SpecialCharTok{\%\textgreater{}\%}
        \FunctionTok{select}\NormalTok{(}\SpecialCharTok{{-}}\NormalTok{participant) }\SpecialCharTok{\%\textgreater{}\%}
        \FunctionTok{pull}\NormalTok{(x) }\SpecialCharTok{\%\textgreater{}\%}
        \FunctionTok{unique}\NormalTok{()}
    
    \CommentTok{\# running the decoding}
\NormalTok{    decoding\_scores }\OtherTok{\textless{}{-}} \FunctionTok{data.frame}\NormalTok{(}
        \FunctionTok{mne\_decoding}\NormalTok{(}\AttributeTok{X =}\NormalTok{ ppt\_data, }\AttributeTok{labels =}\NormalTok{ labels}\DecValTok{{-}1}\NormalTok{)}
\NormalTok{        ) }\SpecialCharTok{\%\textgreater{}\%}
        \CommentTok{\# computing the average over CV folds}
        \FunctionTok{summarise}\NormalTok{(}\FunctionTok{across}\NormalTok{(}\FunctionTok{where}\NormalTok{(is.numeric), mean) )}
    
    \CommentTok{\# appending to previous results}
\NormalTok{    group\_decoding\_scores }\OtherTok{\textless{}{-}} \FunctionTok{bind\_rows}\NormalTok{(group\_decoding\_scores, decoding\_scores)}
    
\NormalTok{\}}

\CommentTok{\# saving the scores}
\FunctionTok{saveRDS}\NormalTok{(}\AttributeTok{object =}\NormalTok{ group\_decoding\_scores, }\AttributeTok{file =} \StringTok{"results/decoding\_scores.rds"}\NormalTok{)}
\end{Highlighting}
\end{Shaded}

\begin{Shaded}
\begin{Highlighting}[]
\CommentTok{\# importing the decoding scores}
\NormalTok{group\_decoding\_scores }\OtherTok{\textless{}{-}} \FunctionTok{readRDS}\NormalTok{(}\AttributeTok{file =} \StringTok{"results/decoding\_scores.rds"}\NormalTok{)}

\CommentTok{\# extracting the timesteps}
\NormalTok{timesteps }\OtherTok{\textless{}{-}}\NormalTok{ raw\_df }\SpecialCharTok{\%\textgreater{}\%}
    \CommentTok{\# filter(participant == ppt) \%\textgreater{}\%}
    \CommentTok{\# select({-}participant) \%\textgreater{}\%}
    \FunctionTok{pull}\NormalTok{(x) }\SpecialCharTok{\%\textgreater{}\%}
    \FunctionTok{unique}\NormalTok{()}

\CommentTok{\# plotting it}
\NormalTok{group\_decoding\_scores }\SpecialCharTok{\%\textgreater{}\%}
    \FunctionTok{t}\NormalTok{() }\SpecialCharTok{\%\textgreater{}\%}
    \FunctionTok{data.frame}\NormalTok{() }\SpecialCharTok{\%\textgreater{}\%}
    \FunctionTok{mutate}\NormalTok{(}
        \AttributeTok{mean =} \FunctionTok{rowMeans}\NormalTok{(}\FunctionTok{across}\NormalTok{(}\DecValTok{1}\SpecialCharTok{:}\DecValTok{20}\NormalTok{) ),}
        \AttributeTok{se =} \FunctionTok{apply}\NormalTok{(}\FunctionTok{across}\NormalTok{(}\DecValTok{1}\SpecialCharTok{:}\DecValTok{20}\NormalTok{), }\DecValTok{1}\NormalTok{, }\ControlFlowTok{function}\NormalTok{ (x) }\FunctionTok{sd}\NormalTok{(x) }\SpecialCharTok{/} \FunctionTok{sqrt}\NormalTok{(}\FunctionTok{length}\NormalTok{(x) ) )}
\NormalTok{        ) }\SpecialCharTok{\%\textgreater{}\%}
    \FunctionTok{mutate}\NormalTok{(}\AttributeTok{lower =}\NormalTok{ mean }\SpecialCharTok{{-}}\NormalTok{ se, }\AttributeTok{upper =}\NormalTok{ mean }\SpecialCharTok{+}\NormalTok{ se) }\SpecialCharTok{\%\textgreater{}\%}
    \FunctionTok{mutate}\NormalTok{(}\AttributeTok{time =}\NormalTok{ timesteps) }\SpecialCharTok{\%\textgreater{}\%}
    \FunctionTok{ggplot}\NormalTok{(}\FunctionTok{aes}\NormalTok{(}\AttributeTok{x =}\NormalTok{ time, }\AttributeTok{y =}\NormalTok{ mean) ) }\SpecialCharTok{+}
    \FunctionTok{geom\_hline}\NormalTok{(}\AttributeTok{yintercept =} \FloatTok{0.5}\NormalTok{, }\AttributeTok{linetype =} \StringTok{"dashed"}\NormalTok{) }\SpecialCharTok{+}
    \FunctionTok{geom\_ribbon}\NormalTok{(}\FunctionTok{aes}\NormalTok{(}\AttributeTok{ymin =}\NormalTok{ lower, }\AttributeTok{ymax =}\NormalTok{ upper), }\AttributeTok{fill =} \StringTok{"steelblue"}\NormalTok{, }\AttributeTok{alpha =} \FloatTok{0.2}\NormalTok{) }\SpecialCharTok{+}
    \FunctionTok{geom\_line}\NormalTok{(}\AttributeTok{color =} \StringTok{"steelblue"}\NormalTok{, }\AttributeTok{size =} \FloatTok{0.8}\NormalTok{) }\SpecialCharTok{+}
    \FunctionTok{labs}\NormalTok{(}\AttributeTok{x =} \StringTok{"Time (s)"}\NormalTok{, }\AttributeTok{y =} \StringTok{"Decoding accuracy (ROC AUC)"}\NormalTok{)}
\end{Highlighting}
\end{Shaded}

\begin{figure}[H]

\caption{Exemplary group-level average timecourse of binary decoding
accuracy (ROC AUC). Decoding was performed using MNE-Python.}

{\centering \includegraphics[width=0.75\textwidth,height=\textheight]{brms_meeg_files/figure-pdf/unnamed-chunk-2-1.pdf}

}

\end{figure}%

Now, we want to \emph{test} whether the group-level average decoding
accuracy is above chance (i.e., 0.5) at each timestep. We use a similar
GAM/GP as previously, but we replace the \(\mathrm{Normal}\) likelihood
function by a \(\mathrm{Beta}\) one to account for the bounded nature of
AUC values (between 0 and 1).

\begin{Shaded}
\begin{Highlighting}[]
\CommentTok{\# fitting the GAM}
\NormalTok{decoding\_gam }\OtherTok{\textless{}{-}} \FunctionTok{brm}\NormalTok{(}
\NormalTok{    auc }\SpecialCharTok{\textasciitilde{}} \FunctionTok{s}\NormalTok{(time, }\AttributeTok{bs =} \StringTok{"cr"}\NormalTok{, }\AttributeTok{k =} \DecValTok{10}\NormalTok{),}
    \AttributeTok{data =}\NormalTok{ decoding\_data,}
    \AttributeTok{family =} \FunctionTok{Beta}\NormalTok{(),}
    \AttributeTok{iter =} \DecValTok{5000}\NormalTok{,}
    \AttributeTok{chains =} \DecValTok{4}\NormalTok{,}
    \AttributeTok{cores =} \DecValTok{4}\NormalTok{,}
    \AttributeTok{file =} \StringTok{"models/decoding\_gam.rds"}
\NormalTok{    )}
\end{Highlighting}
\end{Shaded}

\begin{Shaded}
\begin{Highlighting}[]
\CommentTok{\# fitting the GP}
\NormalTok{decoding\_gp }\OtherTok{\textless{}{-}} \FunctionTok{brm}\NormalTok{(}
\NormalTok{    auc }\SpecialCharTok{\textasciitilde{}} \FunctionTok{gp}\NormalTok{(time, }\AttributeTok{k =} \DecValTok{20}\NormalTok{),}
    \AttributeTok{data =}\NormalTok{ decoding\_data,}
    \AttributeTok{family =} \FunctionTok{Beta}\NormalTok{(),}
    \AttributeTok{control =} \FunctionTok{list}\NormalTok{(}\AttributeTok{adapt\_delta =} \FloatTok{0.9}\NormalTok{),}
    \AttributeTok{iter =} \DecValTok{5000}\NormalTok{,}
    \AttributeTok{chains =} \DecValTok{4}\NormalTok{,}
    \AttributeTok{cores =} \DecValTok{4}\NormalTok{,}
    \AttributeTok{file =} \StringTok{"models/decoding\_gp.rds"}
\NormalTok{    )}
\end{Highlighting}
\end{Shaded}

\begin{figure}[H]

\caption{Posterior predictions of the GAM (left) and GP (right) fitted
on decoding accuracy over time.}

{\centering \includegraphics[width=1\textwidth,height=\textheight]{brms_meeg_files/figure-pdf/decoding-preds-1.pdf}

}

\end{figure}%

Next, we plot the posterior probability of decoding accuracy being above
chance level (plus some epsilon) (Figure~\ref{fig-decoding-post}).

\begin{figure}[!htb]

\caption{\label{fig-decoding-post}Posterior probability of decoding
accuracy being above chance level according to the GAM (left) or the GP
(right).}

\centering{

\includegraphics[width=1\textwidth,height=\textheight]{brms_meeg_files/figure-pdf/fig-decoding-post-1.pdf}

}

\end{figure}%

\begin{verbatim}
  cluster_onset cluster_offset
1     0.1263598      0.6937238
  cluster_onset cluster_offset
1     0.1615063      0.6786611
\end{verbatim}

\begin{figure}[!htb]

\caption{\label{fig-decoding-ratio}Ratio of posterior probabilities of
decoding accuracy being above chance level according to the GAM (left)
or the GP (right).}

\centering{

\includegraphics[width=1\textwidth,height=\textheight]{brms_meeg_files/figure-pdf/fig-decoding-ratio-1.pdf}

}

\end{figure}%

\newpage

\subsection{Application to 2D decoding results (cross-temporal
generalisation)}\label{application-to-2d-decoding-results-cross-temporal-generalisation}

Assume we have M/EEG data and we have conducted cross-temporal
generalisation analyses (\citeproc{ref-king2014}{King \& Dehaene,
2014}). As a result, we have a 2D matrix where each element contains the
decoding accuracy (e.g., ROC AUC) of a classifier trained at timestep
\(\text{training}_{i}\) and tested at timestep \(\text{testing}_{j}\)
(Figure~\ref{fig-sim-timegen}).

\begin{figure}[!htb]

\caption{\label{fig-sim-timegen}Exemplary (simulated) group-level
average cross-temporal generalisation matrix of decoding performance
(ROC AUC).}

\centering{

\includegraphics[width=1\textwidth,height=\textheight]{brms_meeg_files/figure-pdf/fig-sim-timegen-1.pdf}

}

\end{figure}%

Now, we want to test whether and when decoding performance is above
chance level (0.5 for a binary decoding task). These two models are
computationally heavier to fit (more observations and 2D smooth
functions)\ldots{}

\begin{Shaded}
\begin{Highlighting}[]
\CommentTok{\# fitting a GAM with two temporal dimensions}
\NormalTok{timegen\_gam }\OtherTok{\textless{}{-}} \FunctionTok{brm}\NormalTok{(}
    \CommentTok{\# 2D thin{-}plate spline (tp) to model smooth interactions between training and testing time}
    \CommentTok{\# auc \textasciitilde{} s(train\_time, test\_time, bs = "tp", k = 10),}
\NormalTok{    auc }\SpecialCharTok{\textasciitilde{}} \FunctionTok{t2}\NormalTok{(train\_time, test\_time, }\AttributeTok{bs =} \StringTok{"tp"}\NormalTok{, }\AttributeTok{k =} \DecValTok{10}\NormalTok{),}
    \AttributeTok{data =}\NormalTok{ timegen\_data,}
    \AttributeTok{family =} \FunctionTok{Beta}\NormalTok{(),}
    \AttributeTok{iter =} \DecValTok{2000}\NormalTok{,}
    \AttributeTok{chains =} \DecValTok{4}\NormalTok{,}
    \AttributeTok{cores =} \DecValTok{4}\NormalTok{,}
    \AttributeTok{file =} \StringTok{"models/timegen\_gam\_t2.rds"}
\NormalTok{    )}

\CommentTok{\# fitting a GP with two temporal dimensions}
\NormalTok{timegen\_gp }\OtherTok{\textless{}{-}} \FunctionTok{brm}\NormalTok{(}
\NormalTok{    auc }\SpecialCharTok{\textasciitilde{}} \FunctionTok{gp}\NormalTok{(train\_time, test\_time, }\AttributeTok{k =} \DecValTok{20}\NormalTok{),}
    \CommentTok{\# data = timegen\_data \%\textgreater{}\% mutate(auc\_c = scale(x = auc, center = TRUE, scale = FALSE) ),}
    \AttributeTok{data =}\NormalTok{ timegen\_data,}
    \AttributeTok{family =} \FunctionTok{Beta}\NormalTok{(),}
    \AttributeTok{control =} \FunctionTok{list}\NormalTok{(}\AttributeTok{adapt\_delta =} \FloatTok{0.95}\NormalTok{, }\AttributeTok{max\_treedepth =} \DecValTok{20}\NormalTok{),}
    \AttributeTok{iter =} \DecValTok{2000}\NormalTok{,}
    \AttributeTok{chains =} \DecValTok{4}\NormalTok{,}
    \AttributeTok{cores =} \DecValTok{4}\NormalTok{,}
    \AttributeTok{file =} \StringTok{"models/timegen\_gp.rds"}
\NormalTok{    )}
\end{Highlighting}
\end{Shaded}

\begin{figure}[H]

\caption{Posterior probability of decoding accuracy being above chance
level (2D GAM).}

{\centering \includegraphics[width=1\textwidth,height=\textheight]{brms_meeg_files/figure-pdf/gam-timegen-post-preds-1.pdf}

}

\end{figure}%

\subsection{Application to actual M/EEG
data}\label{application-to-actual-meeg-data}

Assessing the reliability of the proposed approach (in comparison to
other methods) using some sort of split-half reliability
(\citeproc{ref-rosenblatt2018}{Rosenblatt et al., 2018})?

\subsection{Comparing the identified onsets/offsets to other
approaches}\label{comparing-the-identified-onsetsoffsets-to-other-approaches}

From Rousselet (2024): Five methods for multiple comparison correction
were considered: two FDR methods; two cluster-based methods; and the
maximum statistics. The two FDR methods were BH95 (Benjamini \&
Hochberg, 1995) and BY01 (Benjamini \& Yekutieli, 2001), which were
applied to the permutation p-values using the \texttt{p.adjust} function
in \texttt{R}. The first cluster-based inference was implemented using a
cluster-sum statistic of squared t-values\ldots{}

Comparing performance of the GAM and GP models with:

\begin{itemize}
\tightlist
\item
  uncorrected univariate tests
\item
  univariate tests + Bonferroni
\item
  univariate tests + BH95 (Benjamini-Hochberg, 1995) aka FDR
\item
  univariate tests + BY01 (Benjamini-Yekutieli, 2001)
\item
  cluster-based permutation test (cluster sum)
\item
  cluster-based permutation test (cluster depth)
\item
  threshold-free cluster enhancement
\item
  change point
\end{itemize}

We used the \texttt{R} package \texttt{permuco} v 1.1.3
(\citeproc{ref-permuco}{Frossard \& Renaud, 2021b}) and the \texttt{R}
package \texttt{changepoint} v 2.2.4 (\citeproc{ref-changepoint}{Killick
et al., 2022a})\ldots{}

\begin{figure}[H]

\caption{Timecourse of squared t-values and s-values (continuous measure
of evidence given by -log2(p-value)) with true onset and onset
identified using the raw (uncorrected) p-values or the corrected
p-values (BH, BY, Bonferroni, or Holm).}

{\centering \includegraphics[width=1\textwidth,height=\textheight]{brms_meeg_files/figure-pdf/test-1d-1.pdf}

}

\end{figure}%

\begin{figure}[H]

\caption{T-values timecourse with true onset/offset and onset/offset
identified using the changepoint package (binary segmentation method, in
green).}

{\centering \includegraphics[width=0.75\textwidth,height=\textheight]{brms_meeg_files/figure-pdf/changepoint-1.pdf}

}

\end{figure}%

Now cluster-based permutations using \texttt{MNE-Python}
(\citeproc{ref-gramfort2013}{Gramfort, 2013}) via the \texttt{R} package
\texttt{reticulate} v 1.35.0 (\citeproc{ref-reticulate}{Ushey et al.,
2024})\ldots{}

\begin{Shaded}
\begin{Highlighting}[]
\CommentTok{\# importing the decoding scores}
\NormalTok{p\_values\_df }\OtherTok{\textless{}{-}} \FunctionTok{readRDS}\NormalTok{(}\AttributeTok{file =} \StringTok{"results/mne\_permutation\_decoding\_scores.rds"}\NormalTok{)}

\CommentTok{\# plotting the s{-}values ({-}log2(p{-}values))}
\NormalTok{p\_values\_df }\SpecialCharTok{\%\textgreater{}\%}
    \FunctionTok{mutate}\NormalTok{(}\AttributeTok{sval =} \SpecialCharTok{{-}}\FunctionTok{log2}\NormalTok{(pval) ) }\SpecialCharTok{\%\textgreater{}\%}
    \FunctionTok{ggplot}\NormalTok{(}\FunctionTok{aes}\NormalTok{(}\AttributeTok{x =}\NormalTok{ time, }\AttributeTok{y =}\NormalTok{ sval) ) }\SpecialCharTok{+}
    \FunctionTok{geom\_line}\NormalTok{(}\AttributeTok{linewidth =} \FloatTok{0.5}\NormalTok{) }\SpecialCharTok{+}
    \FunctionTok{geom\_hline}\NormalTok{(}\AttributeTok{yintercept =} \SpecialCharTok{{-}}\FunctionTok{log2}\NormalTok{(}\FloatTok{0.05}\NormalTok{), }\AttributeTok{linetype =} \DecValTok{2}\NormalTok{) }\SpecialCharTok{+}
    \CommentTok{\# geom\_area(position = "identity") +}
    \FunctionTok{labs}\NormalTok{(}
        \AttributeTok{x =} \StringTok{"Time (ms)"}\NormalTok{,}
        \AttributeTok{y =} \StringTok{"{-}log2(p{-}value)"}
\NormalTok{        )}
\end{Highlighting}
\end{Shaded}

\begin{figure}[!htb]

\caption{\label{fig-mne-cluster}Cluster-based permutation tests via
MNE-Python.}

\centering{

\includegraphics[width=0.75\textwidth,height=\textheight]{brms_meeg_files/figure-pdf/fig-mne-cluster-1.pdf}

}

\end{figure}%

\newpage

\section{Results}\label{results}

\ldots{}

\newpage

\section{Discussion}\label{discussion}

\ldots{}

\subsection{Summary of the proposed
approach}\label{summary-of-the-proposed-approach}

\ldots{}

\subsection{Increasing potential
usage}\label{increasing-potential-usage}

Prepare a wrapper \texttt{R} package and show how to call it in
\texttt{Python} and integrate it with \texttt{MNE-Python}
(\citeproc{ref-gramfort2013}{Gramfort, 2013}) pipelines\ldots{}

\subsection{Limitations and future
directions}\label{limitations-and-future-directions}

\ldots{}

\subsection{Conclusions}\label{conclusions}

\ldots{}

\newpage

\section{Packages}\label{packages}

We used R version 4.2.3 (\citeproc{ref-base}{R Core Team, 2023}) and the
following R packages: brms v. 2.22.0 (\citeproc{ref-brms2017}{Bürkner,
2017}, \citeproc{ref-brms2018}{2018}, \citeproc{ref-brms2021}{2021}),
changepoint v. 2.2.4 (\citeproc{ref-changepoint2022}{Killick et al.,
2022b}; \citeproc{ref-changepoint2014}{Killick \& Eckley, 2014}),
grateful v. 0.2.10 (\citeproc{ref-grateful}{Rodriguez-Sanchez \&
Jackson, 2023}), knitr v. 1.45 (\citeproc{ref-knitr2014}{Xie, 2014},
\citeproc{ref-knitr2015}{2015}, \citeproc{ref-knitr2023}{2023}),
MetBrewer v. 0.2.0 (\citeproc{ref-MetBrewer}{Mills, 2022}), pakret v.
0.2.2 (\citeproc{ref-pakret}{Gallou, 2024}), patchwork v. 1.2.0
(\citeproc{ref-patchwork}{T. L. Pedersen, 2024}), rmarkdown v. 2.29
(\citeproc{ref-rmarkdown2024}{Allaire et al., 2024};
\citeproc{ref-rmarkdown2018}{Xie et al., 2018},
\citeproc{ref-rmarkdown2020}{2020}), scales v. 1.3.0
(\citeproc{ref-scales}{Wickham et al., 2023}), scico v. 1.5.0
(\citeproc{ref-scico}{T. L. Pedersen \& Crameri, 2023}), tidybayes v.
3.0.6 (\citeproc{ref-tidybayes}{Kay, 2023}), tidyverse v. 2.0.0
(\citeproc{ref-tidyverse}{Wickham et al., 2019}).

\newpage

\section{References}\label{references}

\phantomsection\label{refs}
\begin{CSLReferences}{1}{0}
\bibitem[\citeproctext]{ref-rmarkdown2024}
Allaire, J., Xie, Y., Dervieux, C., McPherson, J., Luraschi, J., Ushey,
K., Atkins, A., Wickham, H., Cheng, J., Chang, W., \& Iannone, R.
(2024). \emph{{rmarkdown}: Dynamic documents for r}.
\url{https://github.com/rstudio/rmarkdown}

\bibitem[\citeproctext]{ref-brms2017}
Bürkner, P.-C. (2017). {brms}: An {R} package for {Bayesian} multilevel
models using {Stan}. \emph{Journal of Statistical Software},
\emph{80}(1), 1--28. \url{https://doi.org/10.18637/jss.v080.i01}

\bibitem[\citeproctext]{ref-brms2018}
Bürkner, P.-C. (2018). Advanced {Bayesian} multilevel modeling with the
{R} package {brms}. \emph{The R Journal}, \emph{10}(1), 395--411.
\url{https://doi.org/10.32614/RJ-2018-017}

\bibitem[\citeproctext]{ref-brms2021}
Bürkner, P.-C. (2021). Bayesian item response modeling in {R} with
{brms} and {Stan}. \emph{Journal of Statistical Software},
\emph{100}(5), 1--54. \url{https://doi.org/10.18637/jss.v100.i05}

\bibitem[\citeproctext]{ref-combrisson_exceeding_2015}
Combrisson, E., \& Jerbi, K. (2015). Exceeding chance level by chance:
{The} caveat of theoretical chance levels in brain signal classification
and statistical assessment of decoding accuracy. \emph{Journal of
Neuroscience Methods}, \emph{250}, 126--136.
\url{https://doi.org/10.1016/j.jneumeth.2015.01.010}

\bibitem[\citeproctext]{ref-ehinger_unfold_2019}
Ehinger, B. V., \& Dimigen, O. (2019). Unfold: An integrated toolbox for
overlap correction, non-linear modeling, and regression-based {EEG}
analysis. \emph{PeerJ}, \emph{7}, e7838.
\url{https://doi.org/10.7717/peerj.7838}

\bibitem[\citeproctext]{ref-eklund2016}
Eklund, A., Nichols, T. E., \& Knutsson, H. (2016). Cluster failure: Why
fMRI inferences for spatial extent have inflated false-positive rates.
\emph{Proceedings of the National Academy of Sciences}, \emph{113}(28),
7900--7905. \url{https://doi.org/10.1073/pnas.1602413113}

\bibitem[\citeproctext]{ref-frossard2021}
Frossard, J., \& Renaud, O. (2021a). Permutation Tests for Regression,
ANOVA, and Comparison of Signals: The {\textbf{permuco}} Package.
\emph{Journal of Statistical Software}, \emph{99}(15).
\url{https://doi.org/10.18637/jss.v099.i15}

\bibitem[\citeproctext]{ref-permuco}
Frossard, J., \& Renaud, O. (2021b). Permutation tests for regression,
{ANOVA}, and comparison of signals: The {permuco} package. \emph{Journal
of Statistical Software}, \emph{99}(15), 1--32.
\url{https://doi.org/10.18637/jss.v099.i15}

\bibitem[\citeproctext]{ref-frossard2022}
Frossard, J., \& Renaud, O. (2022). The cluster depth tests: Toward
point-wise strong control of the family-wise error rate in massively
univariate tests with application to M/EEG. \emph{NeuroImage},
\emph{247}, 118824.
\url{https://doi.org/10.1016/j.neuroimage.2021.118824}

\bibitem[\citeproctext]{ref-pakret}
Gallou, A. (2024). \emph{{pakret}: Cite {``{R}''} packages on the fly in
{``{R Markdown}''} and {``{Quarto}''}}.
\url{https://CRAN.R-project.org/package=pakret}

\bibitem[\citeproctext]{ref-gramfort2013}
Gramfort, A. (2013). MEG and EEG data analysis with MNE-python.
\emph{Frontiers in Neuroscience}, \emph{7}.
\url{https://doi.org/10.3389/fnins.2013.00267}

\bibitem[\citeproctext]{ref-hayasaka_validating_2003}
Hayasaka, S. (2003). Validating cluster size inference: Random field and
permutation methods. \emph{NeuroImage}, \emph{20}(4), 2343--2356.
\url{https://doi.org/10.1016/j.neuroimage.2003.08.003}

\bibitem[\citeproctext]{ref-tidybayes}
Kay, M. (2023). \emph{{tidybayes}: Tidy data and geoms for {Bayesian}
models}. \url{https://doi.org/10.5281/zenodo.1308151}

\bibitem[\citeproctext]{ref-changepoint2014}
Killick, R., \& Eckley, I. A. (2014). {changepoint}: An {R} package for
changepoint analysis. \emph{Journal of Statistical Software},
\emph{58}(3), 1--19.
\url{https://www.jstatsoft.org/article/view/v058i03}

\bibitem[\citeproctext]{ref-changepoint}
Killick, R., Haynes, K., \& Eckley, I. A. (2022a). \emph{{changepoint}:
An {R} package for changepoint analysis}.
\url{https://CRAN.R-project.org/package=changepoint}

\bibitem[\citeproctext]{ref-changepoint2022}
Killick, R., Haynes, K., \& Eckley, I. A. (2022b). \emph{{changepoint}:
An {R} package for changepoint analysis}.
\url{https://CRAN.R-project.org/package=changepoint}

\bibitem[\citeproctext]{ref-king2014}
King, J.-R., \& Dehaene, S. (2014). Characterizing the dynamics of
mental representations: the temporal generalization method. \emph{Trends
in Cognitive Sciences}, \emph{18}(4), 203--210.
\url{https://doi.org/10.1016/j.tics.2014.01.002}

\bibitem[\citeproctext]{ref-luck_how_2017}
Luck, S. J., \& Gaspelin, N. (2017). How to get statistically
significant effects in any {ERP} experiment (and why you shouldn't).
\emph{Psychophysiology}, \emph{54}(1), 146--157.
\url{https://doi.org/10.1111/psyp.12639}

\bibitem[\citeproctext]{ref-maris2011}
Maris, E. (2011). Statistical testing in electrophysiological studies.
\emph{Psychophysiology}, \emph{49}(4), 549--565.
\url{https://doi.org/10.1111/j.1469-8986.2011.01320.x}

\bibitem[\citeproctext]{ref-maris2007}
Maris, E., \& Oostenveld, R. (2007). Nonparametric statistical testing
of EEG- and MEG-data. \emph{Journal of Neuroscience Methods},
\emph{164}(1), 177--190.
\url{https://doi.org/10.1016/j.jneumeth.2007.03.024}

\bibitem[\citeproctext]{ref-MetBrewer}
Mills, B. R. (2022). \emph{{MetBrewer}: Color palettes inspired by works
at the metropolitan museum of art}.
\url{https://CRAN.R-project.org/package=MetBrewer}

\bibitem[\citeproctext]{ref-nalborczyk2019}
Nalborczyk, L., Batailler, C., Lœvenbruck, H., Vilain, A., \& Bürkner,
P.-C. (2019). An Introduction to Bayesian Multilevel Models Using brms:
A Case Study of Gender Effects on Vowel Variability in Standard
Indonesian. \emph{Journal of Speech, Language, and Hearing Research},
\emph{62}(5), 1225--1242.
\url{https://doi.org/10.1044/2018_jslhr-s-18-0006}

\bibitem[\citeproctext]{ref-pedersen_hierarchical_2019}
Pedersen, E. J., Miller, D. L., Simpson, G. L., \& Ross, N. (2019).
Hierarchical generalized additive models in ecology: An introduction
with mgcv. \emph{PeerJ}, \emph{7}, e6876.
\url{https://doi.org/10.7717/peerj.6876}

\bibitem[\citeproctext]{ref-patchwork}
Pedersen, T. L. (2024). \emph{{patchwork}: The composer of plots}.
\url{https://CRAN.R-project.org/package=patchwork}

\bibitem[\citeproctext]{ref-scico}
Pedersen, T. L., \& Crameri, F. (2023). \emph{{scico}: Colour palettes
based on the scientific colour-maps}.
\url{https://CRAN.R-project.org/package=scico}

\bibitem[\citeproctext]{ref-pernet2015}
Pernet, C. R., Latinus, M., Nichols, T. E., \& Rousselet, G. A. (2015).
Cluster-based computational methods for mass univariate analyses of
event-related brain potentials/fields: A simulation study. \emph{Journal
of Neuroscience Methods}, \emph{250}, 85--93.
\url{https://doi.org/10.1016/j.jneumeth.2014.08.003}

\bibitem[\citeproctext]{ref-base}
R Core Team. (2023). \emph{{R}: A language and environment for
statistical computing}. R Foundation for Statistical Computing.
\url{https://www.R-project.org/}

\bibitem[\citeproctext]{ref-rasmussen2005}
Rasmussen, C. E., \& Williams, C. K. I. (2005). \emph{Gaussian Processes
for Machine Learning}.
\url{https://doi.org/10.7551/mitpress/3206.001.0001}

\bibitem[\citeproctext]{ref-riutort-mayol_practical_2023}
Riutort-Mayol, G., Bürkner, P.-C., Andersen, M. R., Solin, A., \&
Vehtari, A. (2023). Practical {Hilbert} space approximate {Bayesian
Gaussian} processes for probabilistic programming. \emph{Statistics and
Computing}, \emph{33}(1), 17.
\url{https://doi.org/10.1007/s11222-022-10167-2}

\bibitem[\citeproctext]{ref-grateful}
Rodriguez-Sanchez, F., \& Jackson, C. P. (2023). \emph{{grateful}:
Facilitate citation of r packages}.
\url{https://pakillo.github.io/grateful/}

\bibitem[\citeproctext]{ref-rosenblatt2018}
Rosenblatt, J. D., Finos, L., Weeda, W. D., Solari, A., \& Goeman, J. J.
(2018). All-Resolutions Inference for brain imaging. \emph{NeuroImage},
\emph{181}, 786--796.
\url{https://doi.org/10.1016/j.neuroimage.2018.07.060}

\bibitem[\citeproctext]{ref-rousselet_using_2025}
Rousselet, G. A. (2025). Using cluster-based permutation tests to
estimate {MEG}/{EEG} onsets: {How} bad is it? \emph{European Journal of
Neuroscience}, \emph{61}(1), e16618.
\url{https://doi.org/10.1111/ejn.16618}

\bibitem[\citeproctext]{ref-sassenhagen2019}
Sassenhagen, J., \& Draschkow, D. (2019). Cluster{-}based permutation
tests of MEG/EEG data do not establish significance of effect latency or
location. \emph{Psychophysiology}, \emph{56}(6).
\url{https://doi.org/10.1111/psyp.13335}

\bibitem[\citeproctext]{ref-skukies_modelling_2021}
Skukies, R., \& Ehinger, B. (2021). Modelling event duration and overlap
during {EEG} analysis. \emph{Journal of Vision}, \emph{21}(9), 2037.
\url{https://doi.org/10.1167/jov.21.9.2037}

\bibitem[\citeproctext]{ref-skukies_brain_2024}
Skukies, R., Schepers, J., \& Ehinger, B. (2024, December 9).
\emph{Brain responses vary in duration - modeling strategies and
challenges}. \url{https://doi.org/10.1101/2024.12.05.626938}

\bibitem[\citeproctext]{ref-smith2009}
Smith, S., \& Nichols, T. (2009). Threshold-free cluster enhancement:
Addressing problems of smoothing, threshold dependence and localisation
in cluster inference. \emph{NeuroImage}, \emph{44}(1), 83--98.
\url{https://doi.org/10.1016/j.neuroimage.2008.03.061}

\bibitem[\citeproctext]{ref-teichmann2022}
Teichmann, L. (2022). An empirically driven guide on using bayes factors
for m/EEG decoding. \emph{Aperture Neuro}, \emph{2}, 1--10.
\url{https://doi.org/10.52294/apertureneuro.2022.2.maoc6465}

\bibitem[\citeproctext]{ref-reticulate}
Ushey, K., Allaire, J., \& Tang, Y. (2024). \emph{Reticulate: Interface
to 'python'}. \url{https://CRAN.R-project.org/package=reticulate}

\bibitem[\citeproctext]{ref-tidyverse}
Wickham, H., Averick, M., Bryan, J., Chang, W., McGowan, L. D.,
François, R., Grolemund, G., Hayes, A., Henry, L., Hester, J., Kuhn, M.,
Pedersen, T. L., Miller, E., Bache, S. M., Müller, K., Ooms, J.,
Robinson, D., Seidel, D. P., Spinu, V., \ldots{} Yutani, H. (2019).
Welcome to the {tidyverse}. \emph{Journal of Open Source Software},
\emph{4}(43), 1686. \url{https://doi.org/10.21105/joss.01686}

\bibitem[\citeproctext]{ref-scales}
Wickham, H., Pedersen, T. L., \& Seidel, D. (2023). \emph{{scales}:
Scale functions for visualization}.
\url{https://CRAN.R-project.org/package=scales}

\bibitem[\citeproctext]{ref-wood2003}
Wood, S. N. (2003). Thin Plate Regression Splines. \emph{Journal of the
Royal Statistical Society Series B: Statistical Methodology},
\emph{65}(1), 95--114. \url{https://doi.org/10.1111/1467-9868.00374}

\bibitem[\citeproctext]{ref-wood2004}
Wood, S. N. (2004). Stable and Efficient Multiple Smoothing Parameter
Estimation for Generalized Additive Models. \emph{Journal of the
American Statistical Association}, \emph{99}(467), 673--686.
\url{https://doi.org/10.1198/016214504000000980}

\bibitem[\citeproctext]{ref-knitr2014}
Xie, Y. (2014). {knitr}: A comprehensive tool for reproducible research
in {R}. In V. Stodden, F. Leisch, \& R. D. Peng (Eds.),
\emph{Implementing reproducible computational research}. Chapman;
Hall/CRC.

\bibitem[\citeproctext]{ref-knitr2015}
Xie, Y. (2015). \emph{Dynamic documents with {R} and knitr} (2nd ed.).
Chapman; Hall/CRC. \url{https://yihui.org/knitr/}

\bibitem[\citeproctext]{ref-knitr2023}
Xie, Y. (2023). \emph{{knitr}: A general-purpose package for dynamic
report generation in r}. \url{https://yihui.org/knitr/}

\bibitem[\citeproctext]{ref-rmarkdown2018}
Xie, Y., Allaire, J. J., \& Grolemund, G. (2018). \emph{R markdown: The
definitive guide}. Chapman; Hall/CRC.
\url{https://bookdown.org/yihui/rmarkdown}

\bibitem[\citeproctext]{ref-rmarkdown2020}
Xie, Y., Dervieux, C., \& Riederer, E. (2020). \emph{R markdown
cookbook}. Chapman; Hall/CRC.
\url{https://bookdown.org/yihui/rmarkdown-cookbook}

\bibitem[\citeproctext]{ref-yeung2004}
Yeung, N., Bogacz, R., Holroyd, C. B., \& Cohen, J. D. (2004). Detection
of synchronized oscillations in the electroencephalogram: An evaluation
of methods. \emph{Psychophysiology}, \emph{41}(6), 822--832.
\url{https://doi.org/10.1111/j.1469-8986.2004.00239.x}

\end{CSLReferences}

\newpage

\appendix

\section{Mathematical formulation of the bivariate
GAM}\label{mathematical-formulation-of-the-bivariate-gam}

To model cross-temporal generalisation matrices of decoding performance
(ROC AUC), we extended the initial (decoding) GAM to take into account
the bivariate temporal distribution of AUC values, thus producing
naturally smoothed estimates (timecourses) of AUC values and posterior
probabilities. This model can be written as follows:

\[
\begin{aligned}
\text{AUC}_{i} &\sim \mathrm{Beta}(\mu_{i}, \phi)\\
g(\mu_{i}) &= f \left(\text{train}_{i}, \text{test}_{i} \right)\\
\end{aligned}
\]

where we assume that AUC values come from a \(\mathrm{Beta}\)
distribution with two parameters \(\mu\) and \(\phi\). We can think of
\(f \left(\text{train}_{i}, \text{test}_{i} \right)\) as a surface (a
smooth function of two variables) that we can model using a
2-dimensional splines. Let
\(\mathbf{s}_{i} = \left(\text{train}_{i}, \text{test}_{i} \right)\) be
some pair of training and testing samples, and let
\(\mathbf{k}_{m} = \left(\text{train}_{m}, \text{test}_{m} \right)\)
denote the \(m^{\text{th}}\) knot in the domain of \(\text{train}_{i}\)
and \(\text{test}_{i}\). We can then express the smooth function as:

\[
f \left(\text{train}_{i}, \text{test}_{i} \right) = \alpha + \sum_{m=1}^M \beta_{m} b_{m} \left(\tilde{s}_{i}, \tilde{k}_{m} \right)
\]

Note that \(b_{m}(,)\) is a basis function that maps
\(R \times R \rightarrow R\). A popular bivariate basis function uses
\emph{thin-plate splines}, which extend to
\(\mathbf{s}_{i} \in \mathbb{R}^{d}\) and \(\partial l_{g}\) penalties.
These splines are designed to interpolate and approximate smooth
surfaces over two dimensions (hence the ``bivariate'' term). For \(d=2\)
dimensions and \(l=2\) (smoothness penalty involving second order
derivative):

\[
f \left(\tilde{s}_{i} \right) = \alpha + \beta_{1} x_{i} + \beta_{2} z_{i} +\sum_{m=1}^{M} \beta_{2+m} b_m\left(\tilde{s}_i, \tilde{k}_m\right)
\]

using the the radial basis function given by:

\[
b_m\left(\tilde{s}_i, \tilde{k}_m\right)=\left\|\tilde{s}_i-\tilde{k}_m\right\|^2 \log \left\|\tilde{s}_i-\tilde{k}_m\right\|
\]

where \(\left\|\mathbf{s}_i-\mathbf{k}_{m}\right\|\) is the Euclidean
distance between the covariate \(\mathbf{s}_{i}\) and the knot location
\(\mathbf{k}_{m}\).

\newpage

\section{Threshold-free cluster
enhancement}\label{threshold-free-cluster-enhancement}

Cluster-based permutation approaches require defining a cluster-forming
threshold (e.g., a t- or f-value) as the initial step of the algorithm.
As different cluster-forming thresholds lead to clusters with different
spatial or temporal extent, this threshold modulates the sensitivity of
the subsequent permutation test. The threshold-free cluster enhancement
method (TFCE) was introduced by Smith \& Nichols
(\citeproc{ref-smith2009}{2009}) to overcome this arbitrary threshold.

In brief, the TFCE method works as follows. Instead of picking an
arbitrary cluster-forming threshold (e.g., \(t=2\)), we try all (or
many) possible thresholds in a given range and check whether a given
timestep/voxel belongs to a significant cluster under any of the set of
thresholds\ldots{} Then, instead of using cluster mass, we use a
weighted average between the cluster extend (\(e\), how broad is the
cluster, that is, how many connected samples it contains) and the
cluster height (\(h\), how high is the cluster, that is, how large is
the test statistic) according to the formula:

\[
\text{TFCE} = \int_{h} e(h)^{E} h^{H} \mathrm{d}h
\]

Where\ldots{} the parameters \(E\) and \(H\) are set a priori and
control the influence of the extend and height on the TFCE. Then,
p-value for timestep/voxel \(i\) is computed by comparing it TFCE with
the null distribution of TFCE values. For each permuted signal, we keep
the maximal value over the whole signal for the null distribution of the
TFCE\ldots. But see Sassenhagen \& Draschkow
(\citeproc{ref-sassenhagen2019}{2019})\ldots{}

\newpage

\section{\texorpdfstring{Using the \texttt{R} package and integration
with
\texttt{MNE-Python}}{Using the R package and integration with MNE-Python}}\label{using-the-r-package-and-integration-with-mne-python}

Explain how to use the \texttt{R} package and to integrate it with
\texttt{MNE} epochs\ldots{}

\begin{Shaded}
\begin{Highlighting}[]
\CommentTok{\# to{-}do adding some code here...}
\end{Highlighting}
\end{Shaded}







\end{document}
